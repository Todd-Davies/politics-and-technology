\section*{Overview}

The module introduces the fundamentals of comparative analysis of
political systems. It focuses on the main concepts, theories and
methods of comparative analysis of political systems. Different
political system types, and particularly democracies, will be
systematically presented and analyzed comparatively in terms of their
structures and institutions (polity), actors and processes (politics)
and policies and policy content (policies). The content emphasized is
thus: the methods of comparison, system theory, comparative political
regimes, majority and consensual democracies, parliamentary and
presidential democracies, veto players theory, electoral and party
systems, political values and attitudes, interest group and social
movement influence, and policy analysis in selected policy areas.

\textbf{Note} that this document was intended as a cram-sheet summary
of the course for last-minute reading. It didn't really turn out that
way, and now it's a document for learning/revision, with last-minute
quiz questions in the margin.

\section*{Attribution}

These notes are based off of the readings and lecture slides from the
course. I was unable to include links to the relevant readings or
lecture slides, but if you look on Moodle they should be fairly easy
to find.

\section*{Contribution}

Pull requests are very welcome:
\url{https://git.sr.ht/~todd/University/tree/master/POL10500/review-notes}
