\documentclass[dvipsnames]{minimal}

\usepackage{tikz}
\usepackage[pdftex,active,tightpage]{preview}
\PreviewEnvironment{tikzpicture}

\usetikzlibrary{matrix,calc}

\tikzset{
  mymx/.style={
    matrix of nodes, % this option is available after loading the matrix library
    nodes=block,
    row sep=2cm, % the amount of space separating the nodes vertically
    column sep=3cm, % ^same, but horizontally
  },
  block/.style={
    rectangle,
    draw,
    fill=blue!20,
    rounded corners,
    text width=6em,
    align=center,
    minimum height=4em
  },
  lbl/.style={
    above,
    sloped, % make the text follow the path
    execute at begin node={$}, % begin math mode, for the minus signs etc.
    execute at end node={$}, % end math mode
  }
}
\begin{document}
\begin{tikzpicture}
  \matrix[mymx] (mx) {
    Construct A & Construct B \\
    Independent Variable & Dependent Variable \\
  };
  \path[-latex,every node/.style=lbl]
    (mx-1-1) edge node {Proposition} (mx-1-2)
    (mx-1-1) edge node {}            (mx-2-1)
    (mx-1-2) edge node {}            (mx-2-2)
    (mx-2-1) edge node {Hypothesis}  (mx-2-2)
    ($(mx-1-1)!.5!(mx-2-1)-(5,0)$) edge[-,dashed] ($(mx-1-2)!.5!(mx-2-2)+(2.5,0)$)
    ;
  % to put a node in the middle of something, using the calc library:
  \begin{scope}[every node/.style={execute at begin node=\strut}]
    \node[anchor=east] at ($(mx-1-1.west)-(1,0)$) {Theoretical plane};
    \node[anchor=east] at ($(mx-2-1.west)-(1,0)$) {Empirical plane};
  \end{scope}
\end{tikzpicture}
\end{document}