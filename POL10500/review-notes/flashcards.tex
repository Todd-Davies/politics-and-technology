
\documentclass[frontgrid]{flacards}
\usepackage{tabularx}
\usepackage{color}

\geometry{landscape}

\definecolor{light-gray}{gray}{0.75}

\begin{document}

\pagesetup{3}{4}

\card{
  Science
}{
  Is the systematic and organized body of knowledge in any area of inquiry that is acquired using the ``scientific method''.
}
\card{
  Natural Sciences
}{
  Are the sciences of naturally occurring objects or phenomena, such as light, objects, matter, earth, etc. Natural Sciences can be further broken down into physical sciences, earth sciences, life science, etc.
}
\card{
  Social Sciences
}{
  Are the sciences of people or collections of people (such as groups, firms, economies, societies) and their behaviours. These sciences include psychology, sociology and economics.
}
\card{
  (Scientific) Laws
}{
  Are observed patterns of phenomena or behaviour.
}
\card{
  (Scientific) Theories
}{
  Are systematic explanations of the underlying phenomenon or behaviour.
}
\card{
  Inductive Research (theory building)
}{
  Involves the researcher inferring theoretical concepts and patterns from observed data.
}
\card{
  Deductive Research (theory testing)
}{
  Involves the researcher testing concepts and patterns from a known theory using new empirical data.
}
\card{
  Unit of analysis
}{
  Refers to the person, collective, or object that is the target of the investigation.
}
\card{
  Concepts
}{
  Are generalizable properties or characteristics associated with objects, events or people.
}
\card{
  Construct
}{
  A construct is an abstract concept that is specifically chosen (or created) to explain a given phenomenon. Constructs can be unidimensional such as somebody's weight, or multidimensional such as somebody's communication skills.
}
\card{
  Operational definitions
}{
  Are used for scientific research (in place of dictionary definitions), and define constructs in terms of how they will be empirically measured. For example, the operational definition of temperature will explain what unit it will be measured in.
}
\card{
  Variable
}{
  A variable is a measurable representation of an abstract construct.
}
\card{
  Normological Network
}{
  The overall network of relationships between a set of related constructs.
}
\card{
  Proposition
}{
  A tentative and conjectural relationship between constructs that is stated in a declarative form. E.g. ``An increase in student intelligence causes an increase in their academic achievement''. It does not have to be true, but does have to be empirically testable using data.
}
\card{
  Hypothesis
}{
  A hypothesis is the empirical version of a proposition, that is to say that it says something about the relationship between variables, with the dependent and independent variables clearly specified.
}
\card{
  Strong Hypothesis
}{
  Is a hypothesis with its directionality and causality specified (as opposed to a weak hypothesis, which specifies neither).
}
\card{
  Theory
}{
  A theory is a set of systematically interrelated constructs and propositions intended to explain and predict a phenomenon or behaviour of interest, within certain boundary conditions and assumptions.
}
\card{
  Model
}{
  A model is a representation of all or part of a system that is constructed to study that system. While a theory tries to explain a phenomenon, a model tries to represent a phenomenon.
}
\card{
  Paradigm
}{
  A paradigm is a mental model or frame of reference that we use to organize our reasoning and observations. They are often hard to recognize because they are implicit, assumed and taken for granted.
}
\card{
  Positivism
}{
  Says that knowledge creation is restructured to what can be observed and measured, tending towards theories that can be directly tested (and positively confirmed).
}
\card{
  Post-positivism
}{
  Post-positivism argues that one can make reasonable inferences about a phenomenon by combining empirical observations with logical reasoning.
}
\card{
  Ontology
}{
  Refers to our assumptions about how we see the world.
}
\card{
  Epistemology
}{
  Refers to assumptions about the best way to study the world.
}
\card{
  Methodology
}{
  Concerns the ways in which knowledge of the political world is acquired.
}
\card{
  Micro-political analysis
}{
  Examines the political activity of individuals such as respondents in a mass survey or politicians.
}
\card{
  Macro-political analysis
}{
  Focuses on groups of individuals, structures of power, social classes, economic processes, and the interaction of nation states.
}
\card{
  Idiographic explanations
}{
  Are those that explain a single situation or event in idiosyncratic detail, e.g. if you failed an exam because you forgot it, because you arrived late, you panicked, or had a hangover. They are not generalizable.
}
\card{
  Nomothetic explanations
}{
  Are explanations that seek to explain a class of situations or events rather than a specific situation or event. E.g. generally, students can fail exams because they don't spend enough time studying. They are less precise and less complete than idiographic explanations, but are also economical in their explanations and use few variables. Theories are usually nomothetic.
}
\card{
  Occam's razor
}{
  States that among competing explanations that sufficiently explain the observed evidence, the most simple theory is usually the best.
}
\card{
  Grounded theory building
}{
  Is building theories based on observed patterns of events or behaviours. The theory is `grounded' in empirical observations. The researcher must provide a consistent explanation for all the patterns.
}
\card{
  Internal Validity
}{
  Examines whether the observed change in a dependent variable is indeed caused by a corresponding change in the independent variable, and not by other variables. Internal validity requires that the effect happens if the cause happens (covariation), the cause must precede the effect (temporal precedence) and that there is no plausible alternative explanation.
}
\card{
  External Validity
}{
  Refers to whether the observed associations can be generalized from the sample to the population.
}
\card{
  Construct Validity
}{
  Examines how well a given measurement scale is measuring the theoretical construct that it is expected to measure. E.g. if empathy is being measured, it must be asserted that it's not actually compassion being measured.
}
\card{
  Statistical Conclusion Validity
}{
  Examines the extend that conclusions derived using a statistical procedure are valid.
}
\card{
  Conceptualization
}{
  Is the process by which fuzzy and imprecise constructs (concepts) are defined in concrete and precise terms.
}
\card{
  Operationalization
}{
  Refers to the process of developing indicators or items for measuring a construct, which are called variables.
}
\card{
  Rating scales
}{
  Refer to the values that an indicator can take (kind of like the type of a variable in a programming language.
}
\card{
  Grounded theory
}{
  Is an inductive technique of interpreting recorded data about a social phenomena to build theories about it. The interpretations are `grounded in' the observed empirical data.
}
\card{
  Content Analysis
}{
  Is the systematic analysis of the context of a text (e.g. who says what to whom, why, what are the effects, etc).
}
\card{
  Hermeneutic analysis
}{
  A special type of content analysis, where the researcher tries to ``interpret'' the subjective meaning of a given text within its socio-historical context.
}
\card{
  Archival Research
}{
  Which looks at both primary research (newspapers, film clips, speeches, emails, etc), and secondary sources (e.g. books and articles by others).
}
\card{
  Historical Institutionalism
}{
  A focus on how existing institutional structures patterns of behavior, power relationships, and modes of communication among actors.
}
\card{
  Content Analysis
}{
  The search for patterns or meanings of written/spoken materials, e.g. a comparison of Der Tagespiegel und der Frankfurter Allgemeine Zeitung’s front page coverage of climate change.
}
\card{
  Discourse Analysis
}{
  An effort to find meaning in different forms of discourse, such as how `evil' is discussed by political leaders of different countries with different ideologies.
}
\card{
  Interviews, surveys and database research
}{
  Different types of interviews (structured, semi-strucutred, open), oral or written surveys, and databases from a variety of sources such as think tanks.
}
\card{
  Wave of Democracy
}{
  First popularized by Huntington, a wave of democracy refers to a surge of democracy in the world at a time in history.
}
\card{
  Transitology
}{
  The study of ``democratization'' the process of becoming democracies.
}
\card{
  Democracy
}{
  Government by the people.
}
\card{
  Representative Democracy
}{
  Government by the representatives of the people.
}
\card{
  Majoritarian democracy
}{
  Indicates that the majority of people will do the governing, i.e. whichever group is largest. It is characterized by exclusive power and competitive politics.
}
\card{
  Consensus democracy
}{
  Indicates that as many people as possible should be involved in the governing. Here, having a majority is the minimum requirement. It is characterized by inclusivity, bargaining and compromise.
}
\card{
  Federalism
}{
  Guaranteed division of power between the central government and regional governments.
}
\card{
  Turnover test
}{
  How many times has an incumbent government peacefully handed power to another party as a result of a democratic election?
}
\card{
  Westminster Model
}{
  Equivalent to the majoritarian model; the party with the majority of seats forms a government, elected by a first past the post system. Power is concentrated into the hands of the cabinet. In effect, this is usually the party with the most votes.
}
\card{
  Plurality
}{
  Winning the popular vote (e.g. Hilary Clinton in the US election, 2016).
}
\card{
  Plurality method
}{
  Also known as ``first past the post'', the entity with the most votes wins.
}
\card{
  Manufactured Majorities
}{
  ``Majorities that are artificially created by the electoral system out of a mere plurality of the vote'' - Rae, 1967
}
\card{
  Interest Group Corporatism
}{
  Regular meetings take place between representatives of the government, labour unions and employers organizations to seek agreement on socioeconomic policy. Often seen in `party oriented' democracies over `executive oriented' ones. The coordination process of this is called `concertation'.
}
\card{
  Tripartite Pacts
}{
  Are agreements reached through concertation between government, labour unions and employers organizations.
}
\card{
  Pluralism
}{
  Decision making is mostly in the hands of government, but many non-governmental groups use their resources to exert influence (e.g. by lobbying).
}
\card{
  Corporatism
}{
  All members of the economic sector join an `interest group' which participates in policy making. The state has lots of control over these groups and the members in them.
}
\card{
  Constitutionalism
}{
  Is a central concept in democracies; limit the power of government so that it must follow the law. The government upholding the constitution is part of what makes it legitimate.
}
\card{
  Municipal
}{
  Relating to a town or district and its governing body.
}
\card{
  New Institutionalism
}{
  Institutions which concentrate state and socioeconomic power are required for state capacity and autonomy, and for effective policy change.
}
\card{
  Actor centered institutionalism
}{
  Institutions that disperse state power allow more points of access for veto groups to block these points.
}
\card{
  Party System
}{
  A set of parties that interact in patterned ways. There must be at least two parties, there must be some regularity to the distribution of voter support between parties over time, and there must be a continuity of parties making up the system over time.
}
\card{
  Electoral volatility
}{
  The aggregate turnover from one party to another, from one election to the next.
}
\card{
  Ideological voting
}{
  When voters choose a candidate or party on the basis of which best advances their programmatic interests; ideology is a shortcut for that decision.
}
\card{
  Clientelism
}{
  The exchange of goods and services in return for political support, often involving explicit or implicit quid-pro-quo (e.g. in the extreme case, buying voters).
}
\card{
  Personalistic voting
}{
  Votes are driven on the basis of the personal characteristics of candidates. Also known as `personalism'.
}
\card{
  Bounded rationality
}{
  Decision making and rationality of individuals is limited by the information they posses, the cognitive limitations of their minds, and the finite time they have to make a decision.
}
\card{
  Politics
}{
  Power relations, elections, setting the agenda, maneuvering.
}
\card{
  Policy
}{
  Deciding on outcomes, rules and regulations.
}
\card{
  Polity
}{
  The structure of the actors and institutions.
}
\card{
  Governance
}{
  The process and management of all of the above three.
}
\card{
  Civil Society
}{
  Often we talk about things that the government could be doing, but normal people are doing instead. Countries with large civil societies often do well.
}
\card{
  The Silent Revolution
}{
  Was defined by Ronald Ingelhart, and is characterized by a value shift from materialist to `post-materialist' concerns as people become more wealthy.
}
\card{
  Social Movements
}{
  Are collective, organized and sustained movements that exist outside of normal society, and are aimed at challenging cultural beliefs and practices, or a political or social practice.
}
\card{
  Interest Groups
}{
  Also known as Advocacy Groups are any organizations that seek to influence government policy, but not to actually govern. They are not political parties, but they do try to influence political parties.
}
\card{
  Lobbying
}{
  A strategy by which organized interests seek to influence passage of legislation by exerting direct pressure on members of the legislature.
}
\card{
  Direct lobbying
}{
  Is when groups meet with officeholders or bureaucrats and ask government to change in line with the lobby group's goals. They might help draft legislation, do research to help sway public opinion, or appear in hearings and give their expertise.
}
\card{
  Grassroots lobbying
}{
  Is when interest group members directly lobby for their group by sending letters, making telephone calls, or participating in protests.
}
\card{
  Public interest groups
}{
  Seek a collective good, which will not selectively and materially benefit the membership or activists of the organization.
}
\card{
  Economic Interest Groups
}{
  Have the primary purpose of promoting the financial or business interests of its members.
}
\card{
  Non-Governmental Organization (NGO)
}{
  A legally organized entity created by private persons or organizations with no representation in government.
}
\card{
  Polycentric Governance
}{
  Local governance being networked, creating change from the bottom up.
}
\card{
  Civil society
}{
  Is the space between the private market economy and the public realm of government.
}
\card{
  Civic Engagement
}{
  Active participation in the community (neighborhood associations, sports clubs, cooperatives). The denser these networks, the more likely that members of a community will cooperate for mutual benefit.
}
\card{
  Social Capital
}{
  Was defined by Robert Putnam, and is characterized by high civic engagement and high stocks of social trust, norms and networks that people can draw upon to solve community problems.
}
\card{
  Contentious Politics
}{
  Is the use of disruptive techniques to make a political point, or to change government policy. It occurs when ordinary people join forces in confrontations with elites, authorities and opponents.
}
\card{
  Revolution
}{
  A sudden, fundamental change in power. The thorough replacement of an established government or political system by the people governed.
}
\card{
  Pluralism
}{
  Many actors with the ability to give ideas into the system, and many interest groups; political power is dispersed among them all
}
\card{
  Marxism
}{
  Society is viewed as a composition of socioeconomic classes, based on people's relation to ownership and control of the means of production.
}
\card{
  Rational choice theory
}{
  The individual seeks to maximize personal utility, so how is collective action possible among individuals with a narrow self-interest? Collective action is assumed to be rare in this model, since individuals have little incentive to pursue a public good, and many people choose to free-ride. Selective incentives are used to give restricted benefits to the group.
}
\card{
  Resource mobilization theory
}{
  Emphasizes the importance of group resources, and focuses on things like money, leadership, allies, expertise, etc. It says that social movement activities are not spontaneous and disorganized, and social movement participants aren't irrational.
}
\card{
  New social movement theory
}{
  Is a post-materialist idea that focuses on identity and culture in movement formation and activism. It looks at the collective identity formation, and says that collective action needs to be understood in terms of identity formation.
}
\card{
  Political process model
}{
  What opportunities are available, and how do institutional rules moderate them? When political opportunities arise, contentious politics can be born as people try to realize them.
}
\card{
  Veto Players
}{
  In order to change legislature, a certain number of players must agree to make the proposed change. These are veto players.
}
\card{
  Winset
}{
  Every political system has a set of veto players, and the `winset' is the set of outcomes that will (or can) replace the status quo.
}
\card{
  Political Stability
}{
  If the winset is small (e.g. veto players are ideologically far apart), then changing the status quo becomes difficult, and the system is stable.
}
\card{
  Agenda Setters
}{
  Are veto players that are able to craft ``take it or leave it'' proposals for other veto players. They have significant control over what policies can change the status quo.
}
\card{
  Federation
}{
  Authority is divided between the central state, and local governments.
}
\card{
  Confederation
}{
  Authority is held by independent states and delegated to the central governments.
}
\card{
  Unitary system
}{
  Authority is centrally held with state and local governments administering authority that has been delegated by the central government.
}
\card{
  Dual federalism
}{
  National and state governments are split into their own spheres, and each is supreme in its respective sphere.
}
\card{
  New federalism
}{
  Is an idea in the US to transfer certain powers ceded by states with Roosevelt's New Deal to the federal government back to the states.
}
\card{
  Cooperative federalism
}{
  Is a concept of federalism in which national, state and local governments interact cooperatively and collectively to solve common problems, rather than making policies separately.
}
\card{
  Supranationalism
}{
  Is the idea that autonomous governing bodies have the power and authority to make decisions above the level of member states, and in the interest of the supranational body (e.g. the EU) as a whole.
}
\card{
  Intergovernmentalism
}{
  Is the negotiation process among leaders of national governments inside a supranational body that leads to key supranational decisions.
}
\card{
  Policy Innovation
}{
  The creation of new and novel policies.
}
\card{
  Policy Diffusion
}{
  The idea that policies made at a given place and time are influenced by policy choices made elsewhere. Horizontal diffusion is between governments on the same organizational level, and vertical diffusion is (usually) from lower level governments up to higher level governments.
}
\card{
  Absolute monarchy
}{
  Rulers have absolute power and are defined by their hereditary.
}
\card{
  Totalitarian state
}{
  Authority lies exclusively with the top leadership.
}
\card{
  Fascist state
}{
  Far right ultranationalist and dictorial governments that suppress opposition and strongly regiment society and the military.
}
\card{
  Dictatorship
}{
  Absolute power for the leadership of government.
}
\card{
  Military Juntas
}{
  The military takes over a country, often to `protect democracy'.
}
\card{
  Communist regimes
}{
  A state that tries to realize the communist ideology; to ensure the common ownership of the means of production and remove social classes and money.
}
\card{
  Perestroika
}{
  A shift away from communism towards a market economy.
}
\card{
  Glasnost
}{
  A shift towards open debate.
}
\card{
  Domino Theory
}{
  Is the idea that when a state becomes communist, other nearby states are at risk of becoming communist too.
}
\card{
  Containment Theory
}{
  Aims to influence capitalist states bordering communist states to stop communism from spreading according to domino theory.
}
\card{
  Policy
}{
  A principle, plan, or course of action, as pursued by a government, organization, individual, etc.
}
\card{
  Policy Making
}{
  The act or process of setting and directing the course of action to be pursued by a government, business, etc.
}
\card{
  Satisficing
}{
  Limiting the range of information examined in identifying problems and solutions because information gathering is expensive.
}
\card{
  Bounded rationality
}{
  Decision making and rationality of individuals is limited by the information they posses, the cognitive limitations of their minds, and the finite time they have to make a decision.
}
\card{
  Negotiated decisions
}{
  Compromise, bargaining, accommodation among parties/interests/coalitions in making a decision (decision making is affected by values and preferences of decision makers).
}
\card{
  Agenda Setting
}{
  Is setting the process that determines which issues officials pay serious attention to at any given time.
}
\card{
  Independent Variables
}{
  Are variables that explain other variables.
}
\card{
  Dependent Variables
}{
  Are variables that are explained by other variables.
}
\card{
  Mediating Variables
}{
  Also known as intermediate variables are those that are explained by an independent variable, but also explain a dependent variable
}
\card{
  Moderating Variables
}{
  Influence the relationship between independent and dependent variables.
}
\card{
  Control Variables
}{
  Must be monitored or kept constant during a scientific study.
}
\card{
  Open coding
}{
  Involves the researcher examining raw textual data line-by-line, and identifying discrete events, incidents, ideas, actions, perceptions, etc that are coded as concepts. Each concept is linked to a specific portion of the text for later validation. The technique is called `open' because the researcher is open to finding new concepts in the text.
}
\card{
  Policy Imitation
}{
  Is when one government copies another's successful policies without assessing whether the context in which the policies were successful applies to their own government's situation.
}
\card{
  Policy Coercion
}{
  Is when force (either hard or soft) is applied by one government to another to make it adopt a certain policy. The US does top-down policy coercion when it attaches conditional restrictions to development grants, and the IMF does it when it pushes austerity policies on struggling governments.
}


\end{document}

