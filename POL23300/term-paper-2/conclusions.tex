\section{Conclusion} % 10-15% (300-450 words)

This paper has examined the responses of a select group of corporate
actors within the United States to the issue of anthropogenic global
warming and climate change. It examines the US regulatory environment
inside which corporations operate, the social context in which they
develop policies, and the governance structures they influence.

It has shown how the privatization of climate governance is
increasingly fueled by latent and unfulfilled public demand for action
on environmental issues within the US, which has created an
environment where corporations are increasingly pressured to act, and
produce policies and stances on corporate social responsibility as a
result.

It then drew on the four-featured framework for polycentric governance
outlined by Dorsche and Flaschland, and shows how corporate governance
shows characteristics of each; self-organization (advocacy by business
leaders, grassroots lobbying and corporate peer pressure),
site-specific conditions (purchasing power agreements),
experimentation and learning (corporate policy entrepreneurship and
initiatives to reach carbon negativity), and trust (private
regulators). Specific examples are given to illustrate how each case
can manifest in the private sector, showing the importance of the
corporate sector in facilitating effective environmental governance
within the United States.

The use of private regulation in environmental governance is used to
further demonstrate how corporate actors are able to self-organize,
learn from each other, and build trust in order to facilitate
polycentric governance.

Finally, the paper acknowledges and examines `lagger' corporations,
those that are apathetic at best, or actively work against policies to
tackle global warming at worst. It calls for further research into the
question of how corporate policies on environmental governance are
likely to continue to evolve over time.
