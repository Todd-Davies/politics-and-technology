\section{Introduction} % 10-15% (300-450 words)

In recent years, increasingly urgent reports by the Intergovernmental
Panel on Climate Change (IPCC) have outlined the seriousness of the
issue presented by anthropogenic global warming and the need for
immediate multilateral action across the world. This has led to a
revival of the `environmentalist' social movement, manifesting in
organizations such as the Fridays for Future campaign, and increased
interest from law-making bodies around legislative instruments to curb
the emissions of greenhouse gasses.

As outlined by Harrison, the federal government in the United States
has for years lagged behind other developed countries in the adoption
of effective policies aimed at combating climate change, and remains
the only government in the world to have pulled out of the 2015 Paris
Climate Agreement~\citep{harrison2010united}. As such the demand for
governmental action on climate change within the US is being met at
lower levels of the governance hierarchy, by state and municipal
governments.

The strong emphasis on a market driven economy in the United States
has caused some scholars to examine a phenomenon known as the
``privatization of governance'', which examines the role of
consumer-driven policy shifts and the great agency that corporations
have in creating change through targeted business
practices~\citep{cashore2002legitimacy}.

This paper explores how the framework of polycentric governance
applies to recent trends in corporate governance and environmental
policy inside US corporations. In recognition of the
large shifts in climate policy that took place after COP21 and the
subsequent ratification of the Paris Agreement, this paper focuses its
analysis on the actions of corporations since 2017.

Within these constraints, an argument is constructed showing that
corporations are able to provide effective governance on environmental
issues within their spheres of influence, and as such, can be seen to
fit into the framework of polycentric governance.
