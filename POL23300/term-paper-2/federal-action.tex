\subsection{Climate action in the US federal government}

The US federal government has many veto players due to the fact that
legislation must be approved by the House, the Senate and the
president. The high likelihood of veto, combined with the fact that
individual politicians often vote along the interests of their local
constituents rather than along party lines in order to protect local
businesses from federal climate regulation, means that the US Federal
Government has only ever passed limited climate legislation, which has
often been quickly repealed by later
administrations~\citep{harrison2010united}.

Though the lack of federal action on climate change paints a picture
of apathy in the United States towards climate legislation in favour
of the economy, it belies a large segment of the population who have
consistently felt that climate change is a pressing and urgent concern
above that of business, as illustrated in
Figure~\ref{fig:env-vs-econ}. As such, there has been significant
effort to take action on climate change on a sub-national level, where
the barriers to effective action are not strong. The following
sections will examine climate action at the state level and in the
corporate arena.

\begin{figure}[h!]
  \begin{center}
    \begin{tikzpicture}
      \begin{axis}[
          xlabel=Year,
	  ylabel=Percentage,
          ymin=0,
          ymax=100,
	  title={Public Opinion: environment vs economy},
          align=center,
	  grid=both,
          x tick label style={/pgf/number format/.cd,%
          scaled y ticks = false,
          set thousands separator={},
          fixed},
	  minor grid style={gray!25},
	  major grid style={gray!25},
	  width=0.75\linewidth,
          title style={align=left},
          legend style={at={(0.5,-0.15)},anchor=north},
	  %no marks,
          ]
        \addplot[line width=1pt,solid,color=olive] %
	  table[x=year,y=environmental_protection,col sep=comma]{data.csv};
        \addlegendentry{\% Protection of the environment};
        \addplot[line width=1pt,dotted,color=black] %
          table[x=year,y=economic_growth,col sep=comma]{data.csv};
        \addlegendentry{\% Economic growth};
      \end{axis}
    \end{tikzpicture}
    \caption{American respondents were asked which of these statements
      they most you agreed with; ``\textit{Protection of the
        environment should be given priority, even at the risk of
        curbing economic growth}'' or ``\textit{Economic growth should
        be given priority, even if the environment suffers to some
        extent}``~\citep{gallup}}
    \label{fig:env-vs-econ}
  \end{center}
\end{figure}


% Lots of veto players
% Climate is a political issue in the US, no effective legislation really passed
% Though the largest political player in the US is not acting, others can.

% Cite and draw on 'Energy Transition in Europe and the US: Policy
% Entrepreneurs and Veto Players in Federalist Systems"

% Cite and draw on "The US as an Outlier: Economic and Institutional
% Challenges to US Climate Policy"

\subsection{Climate action in US state governments}

Federalism diffuses the authority of the US government between the
federal government and individual state governments, which leaves room
for state-level to legislation on climate issues. This flexibility has
been put to great use to generate and implement climate legislation in
states such as California, Connecticut\footnote{In Connecticut,
  `Public Act 08-98, An Act Concerning Connecticut Global Warming
  Solutions' requires the state to achieve an 80\% GHG emissions
  reduction from 2001 levels by 2050~\citep{wearestillin-connecticut}.}
and Hawai'i\footnote{Hawai'i has committed to sourcing 100\% of its
  net electricity sales from renewables by
  2045~\citep{wearestillin-hawaii}.}, which can be described as
\textit{compensatory federalism}~\citep{balthasar2019energy}.

US state level action will now be examined in the context of the four
polycentric governance conditions outlined by Dorsch and
Flaschland~\citep{dorsch2017polycentric}.

% self-organization, site-specific conditions, experimentation and
% learning and trust

\subsubsection{Self-organization}

As independent states within the federal system, US states have an
inherent degree of self-organisation as defined to be `\textit{the
  freedom to set up their own rules}'. Though limits exist on what
individual states can legislate on, congressional intervention is
rare\footnote{For instance, in 1967 due to concerns from the
  automobile industry over heterogeneity of state legislation, Congress
  passed an act enabling the federal Environmental Protection Agency
  to override state emission regulations for cars, though California
  was exempted~\citep{david2010environmental}}.

This capability of self-organization and legislative autonomy allows
states to create original solutions to climate issues. This
significant freedom is clearly evident in the pledges made by the
states engaged with the \textit{We Are Still In} movement;
California's pledge alone lists, a Cap-and-Trade programme, the
California Global Warming Solutions Act, and multiple policies around
renewable energy, transportation, efficiency, adaption and resilience,
and natural resources~\citep{wearestillin-california}.

\subsubsection{Experimentation and learning}

\subsubsection{Site-specific conditions}

The more granular nature of state-level legislation allows for a more
tailored approach to climate policies that target the specific context
in which they occur. This contrasts with more broad federal approaches
which by virtue of their scope, cannot take advantage of specific
capabilities of individual actors.

One concern with highly tailored climate legislation is that it could
trade diffusibility for specificity. If a state (i.e. laggard) wishes
to adapt climate policies from a more ambitious state (i.e. leader),
then the complexity of the former having to modify the policies of the
latter to fit its own situation could prove to decrease the likelihood
of horizontal policy diffusion, especially in states with significant
demographic and geophysical differences.

% Race to the top <- site specific condition

% TODO
However, in contrast the site specific conditions present in one state
could enable it to become a leader in legislative terms where...

% Link to the 4 polycentric governance characteristics
% Discuss policy diffusion horizontally and vertically
% Discuss a race to the top among green states

% Cite and draw on:

% "The US as an Outlier: Economic and Institutional Challenges to US
% Climate Policy"
% Importance of state level action

% 'Energy Transition in Europe and the US: Policy Entrepreneurs and
% Veto Players in Federalist Systems"
% Veto players
% Policy reservoir
% Policy diffusion
% Race to the top
% Compensatory federalism

% "Environmental Federalism in the EU and the US"
% Policy diffusion
