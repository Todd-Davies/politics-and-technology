\section{Theoretical framework} % 10% (300 words)

This paper employs the \textit{theory of polycentric governance} as
outlined by Elinor Ostrom as a theoretical framework from which to
base its analysis~\citep{ostrom2009polycentric}. In order identify
polycentric governance happening in the real-world, the classification
developed by Dorsch and Flachsland is used, which identifies four key
features of polycentric governance; \textit{self-organization},
\textit{site-specific conditions}, \textit{experimentation and
  learning} and \textit{trust}~\citep{dorsch2017polycentric}.

The nature of climate change as a pervasive and wicked
problem~\citep{marshall2015don} means that it must be tackled on many
fronts and across many levels of society. Using polycentric governance
theory as an analytical framework enables an analysis spanning all
levels of the decision-making hierarchy, and as such, is used to
illustrate how policy diffusion and policy innovation on the
sub-national level can cause actors such as corporations to help
create a societal shift towards a zero-carbon economy from the bottom
up.

This paper examines the collective behaviour and incentive structures
for corporate actors operating within the United States. Corporate
actors were chosen due to their nature as authoritative decision
makers, as stated by Cutler~\citep{cutler1999private}, their
pervasiveness throughout society, and in order to highlight how their
behaviour as indirect participants in the climate regime fits into the
theory of polycentric governance. Indeed, Ostrom explicitly mentions
private organizations when defining polycentricity: \textit{``A
  polycentric system exists when multiple public and private
  organizations at multiple scales jointly affect collective benefits
  and costs''}~\citep{cole2011global}. As such, the unit of analysis
is defined as a single corporation with the agency to make its own
climate-relevant decisions.

