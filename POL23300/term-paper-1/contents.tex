\section{Introduction}

This paper will look at the United States as global actor regarding
international climate policy, examining the positions it has taken in
international conferences and the internal political circumstances
under which they came about. As such, this paper is split into two
sections; an account of the US’s stance in previous COPs (Kyoto,
Copenhagen and Paris), and an analysis of likely negotiating positions
of US delegations in the upcoming COP25 in Madrid.

\section{International engagement}

The US is a central player in international climate negotiations due to it
having the largest GDP (14.2\% of the global GDP~\cite{countryGdp}) and the second
largest share of GHG emissions (14\% in 2017~\cite{muntean2018fossil}) among attending parties, as well as having
considerable soft power with which it can influence other countries' stances.

Though the US is a hugely important actor in climate negotiations, it
is also an incredibly fickle one owing to the structure of its
democracy, which following the separation of powers principle,
operates government through three branches; the legislative branch,
the executive branch and the judicial branch. The former and the
latter are consensus based and change relatively slowly, but the views
of the executive branch are greatly influenced by the president, who
can be replaced from one day to the next after an election. This means
that the US’ approach to international climate policy can change
dramatically between presidencies.

\subsection{Kyoto}

The Kyoto Protocol was negotiated at COP3 in Kyoto, with Vice President Al Gore playing a crucial role in
striking a last-minute deal~\cite{harrison2010united}. The US wanted flexibility
mechanisms in the deal to make it easier for domestic business to comply, but to get them, it had to find a compromise with the EU. This, along with a negotiation strategy focused on reductions relative to 1990 rather than ones relative to projected emissions in 2010, meant that the US ended up taking deep cuts compared to the `business as usual' case.
On the other hand, the EU got less ambitious targets in return, since its targets dropped to be
more in line with the US ones. This benefitted the EU twice over since
the measurements were now more flexible, and their cuts were lighter too.

Despite the deal being negotiated by Vice President Gore, the COP3
conference was held towards the end of the Clinton presidency, and
Bush moved into office quickly after. The political environment in the
US was very hostile to the Kyoto protocol by this point, with an
administration that heavily deprioritized climate issues, and with the
Byrd-Hagel resolution, which ruled out ratifying any treaty that
didn’t require the same GHG cuts from developing countries as it did
of developed countries ~\cite{harrison2010united}.

This law is indicative of an interesting dynamic of US national
politics, in that unlike other democracies such as in the UK, the main
parties do not hold great sway on the voting behaviour of individual
politicians, who are often highly invested in local interests such as
supporting the local economy and local businesses. The fact that the law passed
through the house is indicative of the fact that many democrats from states with
a heavy fossil fuel or car industry presence, were against climate legislation
on the grounds that it would harm their local economies, despite their party's
official stance being pro-regulation.

\subsection{Copenhagen}

The Copenhagen conference took place in 2009, and was attended by
hundreds of heads of state, including the newly elected President
Obama, keen to present the US as a leader on climate change after the
Bush era~\cite{parker2018climate}. Unfortunately, the Copenhagen conference was a divided
affair, with BASIC countries (Brasil, South Africa, India and China) meeting
together and deciding to veto any binding agreement with specific
targets.

The US decided to support the BASIC group's stance in the hope of getting at a deal
with reduced scope, as opposed to no deal at all. Unfortunately, the deal was so
weak, that it was rejected by many states, and no deal was signed.

\subsection{Paris}

In the years since the Copenhagen accord, the US had continued to
deepen its partnership with China on climate issues, culminating in
climate cooperation announcements that were intended to set the scene
for global action in the COP21 in Paris. The thinking by the Obama
administration, was that if the world’s two biggest emitters were
shown to be prepared to take action well in advance of the conference,
then other delegations would be inspired to do similar~\cite{usPressRelease}.

The US then went on to play a crucial role in facilitating a strong
agreement in Paris; a group of states called the High-Ambition
Coalition were pushing for an ambitious agreement, which was opposed
by developing countries such as China and India. The US elected to
join the High Ambition Coalition, which helped prevent the agreement
being watered down in terms of its mechanisms to ensure
compliance~\cite{parker2018climate}.

With the election of Donald Trump, the US’s international stance swung
wildly again, with the US signaling its intent to pull out of the
Paris agreement (though not pulling out of the UNFCCC). This further serves to
demonstrate the fickleness of US climate policy as a result of a
highly polarised political class, and the fact that many policy decisions are
set directly by the executive.

Of course, as well as pulling out of the Paris agreement, Trump has
also enacted a large swathe of policies aimed at boosting the national
economy at the expense of the country’s climate targets, including
massive investment in fracking for natural gas, reducing funding for
institutions such as the IPCC, NASA and the EPA (which will reduce the capacity
of US institutions to engage with the climate regime), and stopped action to
reduce emissions from the power generation sector~\cite{aei292720}.

\section{Expectations for COP25}

The political climate in the US is highly polarised, with the
Democrats being broadly in favour of climate action, and the
Republicans being broadly against. Since the US currently has a Republican
President, who doesn’t believe in climate change, expectations for the
US are particularly low at this COP. That said, the US remains a
powerful player in negotiations; it has a huge economy which contributes
massively to global GHG emissions, and vast amounts of soft power too.

One interesting aspect of the US as a delegate, is that owing to the
bifurcated views of its political class regarding climate, there will
 a shadow delegation from the US attending COP25 in a non-official
capacity. Similarly to COP23 and COP24, it will be made up of prominent US
politicians and public figures who are keen for the US to take a
leadership role in international climate negotiations, individual
states such as California, cities such as New York, large businesses,
and even US negotiators from previous presidencies when the US was
more amenable to action.

Though unable to negotiate on the behalf of the US, the shadow
delegation represents a large swathe of the US that is prepared to act
in the absence of federal support, a feature of the multi-level and
devolved governance structure of the Government. This serves to
provide motivation to other parties at the conference, showing that the US can still take action on a sub-federal level despite the actions of the executive, and depending on the outcome of the next elections in 2020, the US could
flip back into being a climate leader rather than a climate laggard
again.

\subsection{Negotiation strategies}

Therefore, since both delegations from the US attending COP25 have
different motivations and goals, what would be an appropriate
negotiation strategy for each delegation?

Though there is little appetite for action from the current US
administration, the view of the now departed Rex Tillerson
(Secretary of State, February 2017 to March 2018) continues to hold; that the US should always have
a seat at the table for climate negotiations~\cite{aei292720}. This allows the US
several advantages over and above simply leaving the negotiations
altogether; it’s easier for the US to re-enter negotiations if
political appetite for change increases, by being present at the
negotiations, US diplomats can observe them first hand. In doing so, they can
potentially reinforce the view of the executive that developing
countries should have the same emissions reductions targets as
developed countries (though this is certainly not a majority
viewpoint within UNFCCC members).

A more aggressive and destructive negotiating strategy for the
official delegation could be to curry favour with, and support,
countries that are currently opposed to action such as Saudi Arabia or
Brasil. This would of course be a massive setback for the negotiations
as a whole, but would give the US administration an opportunity to
further support the old fossil-nuclear regime on an international level.

Meanwhile, the US shadow delegation will be unable to participate in
the conference itself, and as such can have little direct influence on
its proceedings. Consequently, their goal could be
to show the extent of support for the climate regime that remains
within the US, despite the highly opposed administration, and possibly
to provide logistical and strategic support to other nations.

It is likely that the shadow delegation will set up a stand outside
the conference, which will aim to emphasise the amount of US support
for climate action, in the face of the administration’s apathy. It
could highlight the fact that the shadow delegation will represent 287 cities and ten states from the US~\cite{wearestillin}. In terms of providing
support, the delegation could, for example, take the form of providing
access to some Obama-era diplomats who could help other negotiating
teams develop effective strategies.

\section{Conclusion}

To summarise, the US has played a variety of roles in climate
negotiations since the inception of the climate regime, with its
appetite for leadership largely determined by the views of the White
House at the time. The upcoming COP25 is unlikely to be any different,
with the incumbent administration unwilling to engage with the climate
regime, yet despite that, there is a significant appetite for
action within the US electorate, as is shown by the strength of the US
shadow delegation that is due to attend in lieu of a strong official
delegation.

\nocite{*}
\addcontentsline{toc}{section}{References}
\bibliographystyle{named}
\bibliography{references}
