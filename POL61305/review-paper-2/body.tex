\section{Political warfare}

Political warfare is defined by Lord to be ``a general category of activities
that includes political action, coercive diplomacy and covert political
warfare'' which often takes place on ideological the level~\cite[p.322,
  34]{robinson2019modern}, and by Galula to be ``when politics becomes an active
instrument of an operation''~\cite{galula2006counterinsurgency}. Examples
include organizing protests (e.g. Russia and Estonia's `Bronze Soldier'
incident~\cite[p.89]{robinson2019modern}), funding of foreign political and
religious organizations (e.g. Iran funding Shi'a militias which often mature
into outright political actors themselves~\cite[p.89]{robinson2019modern}), and
influencing grassroots political opinion aboard (e.g. the use of social media by
ISIS~\cite[p.190]{robinson2019modern}).

\subsection{Electoral intervention}

On particularly pertinent form of political warfare is electoral intervention,
where one party attempts to influence the democratic election process of
another. Conventional methods of electoral intervention revolve around
manipulating electoral registrations \& vote counting (such as occurred in
Zimbabwe by private Israeli firms~\cite{millsafrican}), but more recent methods
aimed at influencing the opposition have emerged, particularly by
Russia~\cite{jones2019russian,carter2018csis}. These methods have now gained the
attention of NATO as a threat of a severity capable of eliciting a
response~\cite{pierini2019election}.

\section{Information warfare and propaganda}

Information warfare in an overloaded term, but here refers to a confrontation in
the information space in order to undermine the political, economic and social
system and effect massive brainwashing of the population, and is known in Russia
as \textit{reflexive control}~\cite{hunter2015challenges}. The idea is that an
opponent should be manipulated into voluntarily choosing an action that is
desired by the aggressor~\cite{thomas2004russia}. It includes denial of
involvement by the aggressor, concealing aggressive actions, obfuscating goals
and retaining plausible deniability and legality~\cite{snegovaya2015putin}.

\section{Conventional warfare}

Conventional warfare is war fought conventional forces (without chemical,
biological or nuclear weapons)~\cite{gortney2010department}, and NATO defines a
'hybrid threat' to include both conventional and non-conventional
means~\cite{jasper2014islamic}. As Galeotti says, hybrid warfare aims to utilize
conventional tools as little as possible, and uses other tactics listed to
ensure that it takes place on the best terms
possible~\cite[p.165]{agnieszka2015ukraine}. In the recent hybrid war between
Russia and Ukraine, `little green men' (soldiers without insignia) were used by
Russia to project military force (backing up Ukrainian separatists who were
officially doing the fighting) with plausible
deniability~\cite{lanoszka2016russian,bekkers2019hybrid}.

\section{Proxy warfare \& Extremism}

Proxy warfare is defined by Mumford as the ``indirect engagement in a conflict
by a third party wishing to influence its strategic
outcome''~\cite{mumford2013proxy}, and as such has a nebulous definition that
could include other tactics listed here depending on what forms of fighting are
taking place (as shown by Marshall's recent examples~\cite{marshall2016civil}).

One notable example is the Iranian Quds Force, which supports proxy groups such
as Hezbollah in the Middle East~\cite{jones2019war, azani2013hybrid}, and is an
active sponsor of terrorism~\cite{byman2008iran}. Terror-like actions are used
extensively by Iran to project power; CSIS describes a recent incident where oil
tankers in the Gulf of Oman were attacked, and outlines the range of possible
threats that Iran can use in the region, from mines to ad-hoc craft with
explosives~\cite{coredsman2018csis}.

\section{Cyber warfare}

Lucas Kello says that cyber warfare is a good fit for hybrid warfare since it
allows aggressors to harm adversaries below the level that would typically
elicit a response. In fact, it is so endemic in the 21st century, that he coined
the term 'unpeace', in recognition of it being so commonly used between hybrid
actors that it is almost constant~\cite{broedersmutually}. Recent examples of
cyber warfare include the Stuxnet worm which targeted Iranian nuclear
reactors~\cite{farwell2011stuxnet}, sabotage of a country's digital health
infrastructure (such as Wanna Cry in 2017~\cite{carlsson2017art}), and Russia's
attacks against Estonia, Georgia and
Ukraine~\cite{hunter2015challenges,miniats2019war,vevera2019dimensions,praks2015hybrid}.

\section{Lawfare}

Lawfare is defined by Dunlap as ``the strategy of using or misusing law as a
substitute for traditional military means to achieve an operational
objective''~\cite{dunlap2008lawfare}, who reasons that lawfare has emerged as an
increasingly viable strategy as international law and globalization have made
the legal judgment of actions more important (especially in democracies, where
support for wars is based on public support).

Kittrie breaks lawfare down into two classes; \textit{instrumental lawfare}
which are legal tools used to achieve the same goals as armed conflict would
(such as asserting territorial claims over land rather than invading it
directly), and and \textit{compliance-leverage disparity lawfare}, which is used
on the kinetic battlefield and gives an advantage to actors who ignore the law
over those that are compelled to follow it (such as by ignoring rules of
engagement)~\cite{kittrie2016lawfare}.

\section{Economic warfare}

Economic warfare is the use of financial and business oriented mechanisms to
harm an adversary, and includes the use of sanctions, destabilisation of energy
prices, transnational crime, preclusive purchasing and other similar
means~\cite{abbott2016understanding,farrar1973preclusive}. These methods can be
tightly aimed at specific targets, but are commonly applied in a manner that
quickly pervades the whole of a society, and can deal psychological damage on a
population-wide basis~\cite{lambert2017brits}.

Most recent usages of economic warfare are associated with China, such as
strategically buying foreign ports to gain leverage over trade and intelligence
insights~\cite{huang2019why}, and subsidizing Huawei in order to gain
marketshare (and thus significant soft and hard power) in the international 5G
market~\cite{sahin2019berlin,davies20205g}.
