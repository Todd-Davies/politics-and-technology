\section{Introduction}

As defined by Weissman, hybrid warfare concerns active measures taken by one
actor towards another actor~\cite{weissmann2019hybrid}. The dynamic nature of
hybrid warfare means that it is not a static repertoire of techniques and
tactics, but rather an ever changing mélange of actions tailored towards a
specific situation against a specific adversary~\cite{schmid2019hybrid}, often
with the intention of shifting the (Clauswitzian~\cite{von1956war}) center of
gravity for the conflict away from the military domain and into other domains.

This literature review aims to list the different tactics that have been used in
hybrid wars. It does so by outlining categories of hybrid tactics and
referencing documented case studies for each. In order to profile the full range
of hybrid tactics, the ``definitional net'' of hybrid warfare is cast
deliberately wide to encompass as much of the phenomenon as possible as
suggested by Wither~\cite{wither2016making}.
