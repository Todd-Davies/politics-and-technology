\section{Categorical Interpretations of 5G}

There are multiple competing and overlapping technological frameworks which can
be used to categorize 5G, plausible selection of which is outlined.
% Emerging technology
Rotolo, Hicks and Martin outline a definition for '\textit{emerging technology}'
based off of five criteria, which serve as a basis to ascertain whether a given
technology such as 5G can be classed as 'emerging'~\cite{rotolo2015emerging}.
% Technology assessment
More broadly, Banta describes the field of \textit{technology
  assessment}~\cite{banta2009technology}, which is aimed at predicting the
longer term outlook of a technology, and this framework is used by the US
Department of Defence to better understand the potential effects of
5G~\cite{csis2020panel}.

% Risk assessment / threat modeling
Related to technology assessment, the EU has undertaken a \textit{risk
  assessment} of 5G in order to better understand the differences between it and
previous versions of mobile networks, and the consequent risks of 5G
adoption~\cite{nis2019eucoordinated}. Various think tanks in the
US~\cite{csis2020twinpillars} and the EU~\cite{kleinhans2019whom}, as well as
UNDIR~\cite{undir2019stemming} having taken this approach, focusing on supply
chain security in particular as summarized by US-based nonprofit
MITRE~\cite{nissen2018deliver}.

% Disruptive innovation + Military innovation
A less convincing interpretation of 5G is that of a '\textit{disruptive
  innovation}' as defined by Christensen~\cite{christensen2015disruptive}
(though competing definitions exist, see Kawamoto and
Spers~\cite{tadao2019systematic}), or as a \textit{military innovation} as
defined by Horowitz and Pindyck~\cite{horowitz2019military}.

\section{Realist conceptualizations}

% What is realism
Many actors consider 5G within a realist framework, where policies arise from
the unregulated competition between states~\cite{donnelly2000realism} and
interactions between states are regarded as zero sum.
% 5G is a sphere of influence to be dominated
The German think tank SWP sees 5G as a contested technological sphere of
influence that is part of a larger rivalry between the US and
China~\cite{rudolf2020sino,lippert2020sinoamerican,lippert2020strategic} with
Europe in the middle and without a strong foreign policy stance of its own
(though able to wield regulatory power to exert some measure of
influence~\cite{bendiek2019europe}).
% Subsidies to gain market control
Another German think tank, DGAP highlights China's use of strategic subsidies to
advantage Huawei at the expense of European suppliers and (indirectly) Germany's
national security and democratic values~\cite{sahin2019berlin}. The NATO
Cooperative Cyber Defence Center of Excellence recently released a paper to the
same effect~\cite{kaska2019huawei}, raising the point that there is precedent
for China to exploit its technological advantages for SIGINT
operations~\cite{demchak2018china}, and describing the opaque ownership
structure of Huawei, which suggests covert state
involvement~\cite{hawes2017transparency,hoffmann2019networks}.

% China's actions are in line with realism, e.g. IP theft
China certainly sees itself as playing an adversarial role in a realist world;
the US has repeatedly leveled the charge of high-tech intellectual property
theft against
China~\cite{drezner2019technological,csis2020panel,csis2019techpolitik,united2018findings}.
Such a viewpoint sees the many geopolitical advantages gained from dominance
over 5G, especially since as many emerging technology scholars have pointed out
new threats caused by new technologies such as 5G can develop over
time~\cite{hauptman2019illuminating,limba2019industry}.

% Banning Huawei could have negative side effects (e.g. segmented markets)
Some realist viewpoints, prominently advocated for by the US, argue for
preventing Chinese infiltration of 5G networks by banning Huawei equipment
altogether~\cite{csis2020panel}. However, risk consultancy Eurasia Group warn
that barring Huawei could have serious effects on the market structure for 5G,
especially if international cooperation is degraded and incompatible standards
emerge~\cite{triolo2018geopolitics}. Furthermore, Lysne outlines the need for
'heterogeneity of suppliers' as critical part of an effective approach for 5G
security, which would be far harder with a highly segmented
market~\cite{lysne2018huawei}.

\section{Neoliberal conceptualizations}
% What is neoliberalism
Neoliberalism aims to build coordination and cooperation among nation states
through international organizations, and takes a more constructive view of how
states can interact~\cite{keohane1986neorealism}.

% The UK is sticking with Huawei but with mitigations
Like the US, the UK acknowledges the threats that China poses in the 5G supply
chain~\cite{uk2019telecoms}, but has responded by working with China to set up a
review board with access to Huawei software and hardware that aims to review it
for exploitable threats~\cite{uk2019huaweiboard} (in line with what the think
tank SNV suggests~\cite{kleinhans20195g}). Chatham House points out that the
UK's existing mobile network infrastructure has a strong path dependence on
5G~\cite{chatham2019whosafraid}, which goes towards explaining this engaged and
somewhat mitigatory approach.

% The EU's approach is more passive towards Huawei
The EU takes a typically non-aggressive and neoliberal approach. It doesn't
reject Huawei but instead gives member states cover to do it of their own accord
through its critical 5G Cybersecurity report~\cite{eu2019cybersecurity}. That
said, the EU's cybersecurity agency notes that supply chain attacks are only one
vector by which 5G could be used for malicious purposes, viewing Huawei's
trustworthiness as an issue less important to the overall picture of 5G
security, but sees 5G as an important and complex threat
overall~\cite{enisa20195gnetworks}.

% China isn't playing the same game as everybody else, the EU should change it's
% strategy to be more assertive & proactive.
The EU is built on the idea that trade helps to reduce conflict, yet scholars
such as Henry Farrell are moving back towards a realist view by showing
how trade can be weaponized to undermine this foundation, and how Huawei is an
example of China has failing to integrate properly into the global neoliberal
economic system~\cite{farrell2019weaponized}. A recent paper from the Carneige
Endowment for International Peace describes how the EU is shifting to a more
aggressive approach in order to maintain it's multilateral approach in this more
adversarial environment~\cite{brattberg2020eu}.

Going forward the German think tanks SWP and SNV suggest the EU could try other
multilateral solutions than trade, such as trying to export its cybersecurity
laws to other places in order to bolster the security and resilience of the
whole ecosystem~\cite{bendiek2019europe, saslow2019globalcyber}.

\section{Misconceptions of 5G}
Recent empirical observations have shown members of the general public to be
intensely skeptical of 5G, with multiple studies describing misinformation
campaigns and conspiracy theories that associate 5G as a causative factor for
the 2020 coronavirus
pandemic~\cite{ahmed2020dangerous,shahsavari2020conspiracy,pummerer2020conspiracy}.
This has had a particular impact in the UK, where mobile network base stations
have been destroyed~\cite{ahmed2020dangerous} and the National Government has
issued advice to citizens that there is no link between the 2020 coronavirus
pandemic and 5G~\cite{uk2020covid}. Yet the scientific consensus does not
pervade all of Government, Glastonbury Town Council recently released a
controversial report by it's 5G Advisory Committee on the dangers and risks of
5G to human health~\cite{glastonbury20205g}.
