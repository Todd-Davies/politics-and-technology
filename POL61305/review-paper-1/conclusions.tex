\section{Conclusions}
In line with the Collingridge dilemma~\cite{collingridge1982social}, the effects
of 5G on society will only be revealed as it is implemented. In the midst of
this uncertainty, this paper has examined the literature on 5G from a range of
actors, examining how 5G is conceptualized and the real world effects it is
having on the behaviour of actors on the world stage.

For nation states, trustworthiness of suppliers is of vital importance since 5G
technology is hard to verify. As such, further research on the relationship
between the market structure of 5G and the trustworthiness of vendors is
important and required in order to explore how states can use collaborative and
economic tools such has competition law to improve the overall health of the 5G
ecosystem.

Similarly, more research is warranted on how 5G is framed by actors and
perceived by ordinary users. While conventional wisdom may consider 5G as a
public technology which contributes to the public good (as defined by
Drezner~\cite{drezner2019technological}), the wealth of misinformation around
5G suggests that it has potential to be understood differently. This is
important, since different disinformation campaigns targeting 5G and other
emerging technologies could be used as a hybrid warfare tactic by malign
actors~\cite{bekkers2019hybrid} in future conflicts.
